%================================= Resumo e Abstract ========================================
\setlength{\absparsep}{18pt} % ajusta o espaçamento dos paragrafos do resumo

% ---
% resumo em português
% ---
\begin{resumo}
Paxos é um mecanismo de replicação ativa que consege manter um mesmo estado compartilhado
entre servidores que atendem a requisições de uma aplicação. É incomum encontrar
apliacações onde a parte principal do processamento acotece através de um algorítimo de
replicação como Paxos devido ao seu custo em termos do número de mensagens trocadas, o que
limita a escalabilidade do sistema para algumas poucas réplicas. Acreditamos que seja
possível utilizar replicação ativa não só como substrato de apoio a coordenação de
aplicações distribuídas de baixo acoplamento, mas também como base para a construção de
uma aplicação completa. Um exemplo que permite replicar a aplicação usando replicação
ativa através do algoritmo Paxos é a biblioteca Treplica que provê uma forma simples e
orientada a objetos a construção de aplicações altamente confiáveis. Utilizando essa
biblioteca, a aplicação resultante preserva a consistência de uma aplicação centralizada e
adiciona a tolerância a falhas de uma aplicação distribuída. Nessa dissertação exploramos
a questão da reconfiguração em sistemas de replicação ativa. Em particular, cobiçamos
transformar Treplica em uma biblioteca reconfigurável. Propomos dois novos mecanismos:
(1) protocolo eficiente para transferência de estado; e (2) adição de novas réplicas sem
aumentar de forma significativa o custo de manutenção da consistência do sistema como um
todo. Nossa estratégia utiliza os dois mecanismos para criação de réplicas leitoras, que
são capazes de atender todas as requisições da aplicação sem no entanto participarem
ativamente das operações custozas do algoritmo Paxos.

\textbf{Palavras-chaves}: Replicação ativa. Paxos. Reconfiguração. Transferência de
estado.

\end{resumo}
% ---


% ---
% resumo em ingl�s
% ---
\begin{resumo}[Abstract]
 \begin{otherlanguage*}{english}

Paxos is an active replication algorithm that keeps the same shared state consistently
among servers that handle requests from an application. It is unusual to find applications
where the main processing happens through a replication algorithm such as Paxos, mostly
due to the high number of exchanged messages required to keep the state consistent. This
restricts the system scalability to a handful of replicas. We believe that is possible to use
active replication not only as a low level support to loosely coupled distributed
applications, but as a framework to fully replicated applications. An example that
provides replicated application using active replication through Paxos is the Treplica
library that provides a simple and object-oriented way to build highly reliable
applications. Using this library, the application preserves the consistency of a
centralized implementation and it adds fault tolerance of a distributed application. In
this paper we explored reconfiguration on systems that use active replication. We proposed
two mechanisms: (1) efficient protocolo to state transfer; and (2) incorporation of new
replicas in the system with no significant increasing in the cost to keep the whole system
consistent. Our approach uses both mechanisms to create reader replicas, capable to answer
all application requests without taking an active part in the costly operations of the
Paxos algorithm.

\textbf{Key-words}: Active replication. Paxos. Reconfiguration. State transfer.

 \end{otherlanguage*}
\end{resumo}
% ---

%===========================================================================================
