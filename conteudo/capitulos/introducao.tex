\chapter*[Introdução]{Introdução}
\addcontentsline{toc}{chapter}{Introdução}

Atualmente, muitos dos programas que estamos acostumados a utilizar no dia-a-dia são
distribuídos. Simples rotinas diárias como ler correio eletrônico ou navegar na Internet
envolvem algum tipo de computação distribuída \cite{cachin11}. Podemos definir sistemas
distribuídos como um conjunto de servidores (físicos ou virtuais) independentes que
apresentam-se a seus usuários como um sistema único e coerente \cite{tanenbaum07}. Também
é verdade que as falhas nos servidores podem ocorrer de maneira independente. Essa noção
de falhas parciais, levou \citeonline{cachin11} a definir que sistemas distribuídos são
sistemas onde a falha de uma máquina que você nem sabia que existia pode tornar sua
própria máquina inutilizável.

Quando existe uma falha em um servidor, o desafio para os que ainda estão operacionais é
manter consistência na sincronização de suas atividades. Ou seja, a colaboração entre os
servidores deve ser suficientemente robusta para suportar falhas parciais \cite{cachin11}.
Sendo assim, o objetivo de sistemas tolerantes a falhas é continuar a prover serviços
mesmo na ocorrência de defeitos em alguns de seus componentes, podendo até levar a
reconfiguração do sistema para exclusão do componente defeituoso \cite{tanenbaum07}.

Uma estratégia amplamente empregada em sistemas distribuídos para prover tolerância a
falhas e aumento na capacidade de processamento é a replicação de dados. A
\emph{replicação ativa} \cite{schneider90} é uma estratégia de replicação voltada para
manutenção de um mesmo estado compartilhado entre servidores que atendem requisições de
uma mesma aplicação, sendo cada um desses servidores chamado de \emph{réplica}. A
replicação ativa é baseada na re-execução em cada uma das réplicas as operações que
alteram o estado compartilhado, devidamente ordenadas por um algoritmo apropriado
\cite{schneider90}. Dentre os vários algoritmos de replicação, um dos mais amplamente
usados e estudados atualmente é o algoritmo Paxos \cite{lamport98}.

Algoritmos de replicação são parte fundamental de várias arquiteturas distribuídas de
software \cite{chandra07, hupfeld08b, maccormick04}, sendo particularmente usados como
soluções para coordenação entre processos que implementam aplicações com garantias de
consistência relaxadas \cite{burrows06} ou fazendo parte de algum esquema hierárquico de
bloqueios \cite{lampson96}. De forma geral, é incomum encontrar aplicações onde a parte
principal do processamento acontece através de replicação ativa devido ao fato que essa
estratégia possui um custo considerável em termos do número de mensagens trocadas, o que
limita a escalabilidade do sistema além de algumas poucas réplicas \cite{lampson96}.

Acreditamos que seja possível utilizar replicação ativa não só como substrato de apoio a
coordenação de aplicações distribuídas de baixo acoplamento, mas também como base para a
construção de uma aplicação completa. Com esse objetivo, a biblioteca de replicação
Treplica \cite{vieira08a, vieira-tr10b} foi projetada para o modelo computacional
\emph{falha-e-recuperação assíncrono} e provém uma forma simples e orientada a objetos a
construção de aplicações altamente confiáveis. Através de uma interface de programação
simples, Treplica permite que o projetista da aplicação pense em termos de operações com
semântica sequencial, similar àquela encontrada em sistemas de processamento transacional.
Utilizando essa biblioteca, ou soluções similares, a aplicação resultante preserva a
consistência de uma aplicação centralizada e adiciona a tolerância a falhas de uma
aplicação distribuída.

Apesar da maior confiabilidade, construir uma aplicação somente com replicação ativa
potencialmente limita o quanto que essa aplicação pode tirar proveito dos ganhos de escala
advindos de ser uma aplicação distribuída. Resultados experimentais mostram o impacto
negativo do aumento da escala no desempenho da implementação de Paxos encontrada em
Treplica \cite{vieira09}. Gostaríamos de ser ser capazes de, não só tornar a capacidade de
processamento proporcional ao número de servidores empregados, mas também de variar essa
capacidade dinamicamente em resposta às mudanças da demanda gerada. Dessa forma, teríamos
aplicações com a simplicidade de programação de aplicações centralizadas e características
de aplicações distribuídas.

Nesse trabalho cobiçamos transformar Treplica em uma biblioteca reconfigurável. O problema
de reconfiguração é complexo, principalmente na presença de falhas e assincronia. Sua
resolução pode ser obtida através de duas estratégias: (1) baseado em transições de visões
do conjunto de réplicas operacionais \cite{birman87a, birman87b}; (2) definição, via
consenso, de uma nova configuração a partir da construção de uma barreira que, quando
alcançada pelas réplicas, faz com que elas abandonem a configuração vigente e ingressem na
nova configuração definida (caso elas façam parte dela) \cite{lamport10}. Em geral,
sistemas de replicação ativa não suportam mecanismo de reconfiguração, sendo definidos
como grupos de processos estáticos \cite{chandra96, lamport98}. Do ponto de vista teórico,
podemos encontrar um tratamento formal do problema~\cite{lamport10}, mas sistemas práticos
tendem a evitar esta questão de forma a simplificar a construção do sistema
\cite{chandra07}.

A estratégia de reconfiguração descrita por \cite{lamport10} pode ser bem complexa, ela
abrange adição e remoção de réplicas, nessa dissertação estamos focados em um sub-conjunto
desse problema onde apenas novas réplicas com estado inicial vazio são adicionadas
consequentemente não precisamos de uma política para tratar réplicas com estado. A nossa
proposta está fundamentada na utilização de \emph{réplicas leitoras}, que são capazes de
atender a todas as requisições da aplicação sem, no entanto, requerer acesso à memória
persistente e sem, na prática, participarem ativamente das operações custosas do algoritmo
Paxos. Identificamos a necessidade da criação de um mecanismo eficiente para
\emph{transferência de estado} entre réplicas. Estamos preocupados com o impacto inerente
para implantar uma nova réplica em um aglomerado em tempo de execução. Tal impacto
dificulta a viabilidade das técnicas de autogestão, porque dependendo do cenário,
adicionar uma nova réplica pode comprometer o desempenho de um sistema sobrecarregado
\cite{vilaca09}. Esse trabalho apresenta duas novas funcionalidades para expansão da
biblioteca Treplica:

\begin{enumerate}
  \item Mecanismo de transferência de estado: criação de um protocolo eficiente para
    transferência de estado entre réplicas.
  \item Réplicas leitoras: possibilidade da utilização de réplicas que não participam do
    processo de decisão de instâncias de consenso.
\end{enumerate}

O restante dessa dissertação se encontra organizada da seguinte forma: O Capítulo 1
introduz a estratégia de replicação ativa para tolerância a falhas, o modelo computacional
utilizado e descreve o algoritmo Paxos e a biblioteca Treplica. Já o Capítulo 2 expõe em
detalhes a ideia de reconfiguração em Treplica e os componentes necessários para execução
de tal tarefa. O Capítulo 3 apresenta e discute os resultados. Por fim, a Conclusão
explicita conclusões e trabalhos futuros.

