\chapter*[Introdução]{Introdução}
\addcontentsline{toc}{chapter}{Introdução}

Atualmente, muitos dos programas que estamos acostumados a utilizar no dia-a-dia são
distribuídos. Simples rotinas diárias como ler correio eletrônico ou navegar na Internet
envolvem algum tipo de computação distribuída \cite{cachin11}. Podemos definir sistemas
distribuídos como um conjunto de servidores (físicos ou virtuais) independentes que
apresentam-se a seus usuários como um sistema único e coerente \cite{tanenbaum07}. Ao
optar pela computação distribuída procuramos alcançar os seguintes benefícios para a
aplicação, descritas por \citeonline{birman05}:

\begin{itemize}
  \item Tolerância a falhas: capacidade dos componentes de um sistema distribuído
    recuperar-se de defeitos sem realizar ações incorretas;
  \item Alta disponibilidade: capacidade de manter a prestação de serviço mesmo durante
    períodos de falhas de alguns servidores. Um sistema altamente disponível pode fornecer
    serviços reduzidos por curtos períodos de tempo enquanto se recupera de falhas;
  \item Capacidade de recuperação: correção de servidores avariados e re-aderência ao
    sistema;
  \item Consistência: capacidade do sistema coordenar ações de múltiplos servidores,
    muitas vezes na presença de concorrência e falhas;
  \item Escalabilidade: capacidade dos recursos de um sistema suportar aumento de carga;
  \item Segurança: capacidade de proteger os dados, serviços e recursos contra o uso
    indevido por usuários não autorizados;
  \item Desempenho previsível: garantia que um sistema distribuído alcance níveis
    desejados de desempenho;
  \item Pontualidade: em sistemas sujeitos a restrições de tempo real, garantia que
    a computação seja executada dentro dos limites de tempo definido.
\end{itemize}

Como consequência da capacidade de execução simultânea de de forma independente, os
servidores de um sistema distribuído podem parar de funcionar também de forma
independente \cite{cachin11}. Devido a essa noção de falhas parciais,
\citeonline{cachin11} definiu que sistemas distribuídos são sistemas onde a falha de uma
máquina que você nem sabia que existia pode tornar sua própria máquina inutilizável.

Replicação de dados é uma estratégia amplamente empregada em sistemas distribuídos para
prover tolerância a falhas e aumentar a capacidade de processamento. A \emph{replicação
ativa}~\cite{schneider90} é uma estratégia de replicação voltada para manutenção de um mesmo
estado compartilhado entre servidores que atendem requisições de uma mesma aplicação,
sendo cada um desses servidores chamados de \emph{réplica}. A replicação ativa é baseada
na reexecução das operações que alteram estado compartilhado, devidamente ordenados por um
algoritmo apropriado~\cite{schneider90}. Dentre os vários algoritmos de replicação, um dos
mais amplamente usados e estudados atualmente é o algoritmo Paxos~\cite{lamport98}.

Algoritmos de replicação são parte fundamental de várias arquiteturas distribuídas de
software~\cite{chandra07,  hupfeld08b, maccormick04}, sendo particularmente usados como
soluções para coordenação entre processos que implementam aplicações com garantias de
consistência relaxada~\cite{burrows06} ou fazendo parte de algum esquema hierárquico de
bloqueios ~\cite{lampson96}. De forma geral, é incomum encontrar aplicações onde a parte
principal do processamento acontece através de replicação ativa devido ao fato que essa
estratégia possui um custo considerável em termos do número de mensagens trocadas, o que
limita a escalabilidade do sistema além de algumas poucas réplicas~\cite{lampson96}.

Acreditamos que seja possível utilizar replicação ativa não só como substrato de apoio a
coordenação de aplicações distribuídas de baixo acoplamento, mas também como base para a
construção de uma aplicação completa. Com esse objetivo, a biblioteca de replicação
Treplica~\cite{vieira08a, vieira-tr10b} foi projetada para prover uma forma simples e
orientada a objetos de se construir aplicações altamente confiáveis. Através de uma
interface de programação simples, Treplica permite que o projetista de aplicação pense em
termos de operações com semântica sequencial, similar àquela encontrada em sistemas de
processamento transacional. Utilizando essa biblioteca, ou soluções similares, a aplicação
resultante preserva a consistência de uma aplicação centralizada e adiciona a tolerância a
falhas de uma aplicação distribuída.

Apesar da maior confiabilidade, construir uma aplicação somente com replicação ativa
potencialmente limita o quanto que essa aplicação pode tirar proveito dos ganhos de escala
advindos de ser uma aplicação distribuída. Resultados experimentais mostram o impacto
negativo do aumento da escala no desempenho da implementação de Paxos encontrado em
Treplica~\cite{vieira09}. Gostaríamos de ser ser capazes de, não só tornar a capacidade de
processamento proporcional ao número de servidores empregados, mas também de variar essa
capacidade dinamicamente em resposta às mudanças da demanda gerada. Dessa forma, teríamos
aplicações com a simplicidade de programação de aplicações centralizadas e características
de aplicações distribuídas.




Atualmente, muitos dos programas que estamos acostumados a utilizar no dia-a-dia são
distribuídos. Simples rotinas diárias como ler correio eletrônico ou navegar na Internet
envolvem algum tipo de computação distribuída \cite{cachin11}. Podemos definir sistemas
distribuídos como um conjunto de servidores (físicos ou virtuais) independentes que
apresentam-se a seus usuários como um sistema único e coerente \cite{tanenbaum07}. Também
é verdade que as falhas nos servidores podem ocorrer de maneira independente. Essa noção
de falhas parciais, levou \citeonline{cachin11} a definir que sistemas distribuídos são
sistemas onde a falha de uma máquina que você nem sabia que existia pode tornar sua
própria máquina inutilizável.

Quando existe uma falha em um servidor, o desafio para os que ainda estão operacionais é
manter consistência na sincronização de suas atividades. Ou seja, a colaboração entre os
servidores deve ser suficientemente robusta para suportar falhas parciais \cite{cachin11}.
Sendo assim, o objetivo de sistemas tolerantes a falhas é continuar a prover serviços
mesmo na ocorrência de defeitos em alguns de seus componentes, podendo até levar a
reconfiguração do sistema para exclusão do componente defeituoso \cite{tanenbaum07}.

Uma estratégia amplamente empregada em sistemas distribuídos para prover tolerância a
falhas e aumento na capacidade de processamento é a replicação de dados. A
\emph{replicação ativa} \cite{schneider90} é uma estratégia de replicação voltada para
manutenção de um mesmo estado compartilhado entre servidores que atendem requisições de
uma mesma aplicação, sendo cada um desses servidores chamado de \emph{réplica}. A
replicação ativa é baseada na re-execução em cada uma das réplicas as operações que
alteram o estado compartilhado, devidamente ordenadas por um algoritmo apropriado
\cite{schneider90}. Dentre os vários algoritmos de replicação, um dos mais amplamente
usados e estudados atualmente é o algoritmo Paxos \cite{lamport98}.

Algoritmos de replicação são parte fundamental de várias arquiteturas distribuídas de
software \cite{chandra07, hupfeld08b, maccormick04}, sendo particularmente usados como
soluções para coordenação entre processos que implementam aplicações com garantias de
consistência relaxadas \cite{burrows06} ou fazendo parte de algum esquema hierárquico de
bloqueios \cite{lampson96}. De forma geral, é incomum encontrar aplicações onde a parte
principal do processamento acontece através de replicação ativa devido ao fato que essa
estratégia possui um custo considerável em termos do número de mensagens trocadas, o que
limita a escalabilidade do sistema além de algumas poucas réplicas \cite{lampson96}.

Acreditamos que seja possível utilizar replicação ativa não só como substrato de apoio a
coordenação de aplicações distribuídas de baixo acoplamento, mas também como base para a
construção de uma aplicação completa. Com esse objetivo, a biblioteca de replicação
Treplica \cite{vieira08a, vieira-tr10b} foi projetada para o modelo computacional
\emph{falha-e-recuperação assíncrono} e provém uma forma simples e orientada a objetos a
construção de aplicações altamente confiáveis. Através de uma interface de programação
simples, Treplica permite que o projetista da aplicação pense em termos de operações com
semântica sequencial, similar àquela encontrada em sistemas de processamento transacional.
Utilizando essa biblioteca, ou soluções similares, a aplicação resultante preserva a
consistência de uma aplicação centralizada e adiciona a tolerância a falhas de uma
aplicação distribuída.

Apesar da maior confiabilidade, construir uma aplicação somente com replicação ativa
potencialmente limita o quanto que essa aplicação pode tirar proveito dos ganhos de escala
advindos de ser uma aplicação distribuída. Resultados experimentais mostram o impacto
negativo do aumento da escala no desempenho da implementação de Paxos encontrada em
Treplica \cite{vieira09}. Gostaríamos de ser ser capazes de, não só tornar a capacidade de
processamento proporcional ao número de servidores empregados, mas também de variar essa
capacidade dinamicamente em resposta às mudanças da demanda gerada. Dessa forma, teríamos
aplicações com a simplicidade de programação de aplicações centralizadas e características
de aplicações distribuídas.

Nesse trabalho cobiçamos transformar Treplica em uma biblioteca reconfigurável. O problema
de reconfiguração é complexo, principalmente na presença de falhas e assincronia. Sua
resolução pode ser obtida através de duas estratégias: (1) baseado em transições de visões
do conjunto de réplicas operacionais \cite{birman87a, birman87b}; (2) definição, via
consenso, de uma nova configuração a partir da construção de uma barreira que, quando
alcançada pelas réplicas, faz com que elas abandonem a configuração vigente e ingressem na
nova configuração definida (caso elas façam parte dela) \cite{lamport10}.

A estratégia de reconfiguração descrita por \cite{lamport10} pode ser bem complexa, ela
abrange adição e remoção de réplicas, nessa dissertação estamos focados em um sub-conjunto
desse problema onde apenas novas réplicas com estado inicial vazio são adicionadas,
consequentemente não precisamos de uma política para tratar réplicas com estado. Em outras
palavras, a política para tratar recuperação de falhas em Treplica não foi alterada.

Nesse trabalho propomos um mecanismo de reconfiguração do algoritmo Paxos que permite a
adição de novas réplicas sem aumentar de forma significativa o custo de manutenção da
consistência do sistema como um todo. A nossa proposta tem como base fundamental o
estabelecimento de \emph{réplicas leitoras}, que são capazes de atender a todas as
requisições de aplicação sem, no entanto, requerer acesso à memória persistente e sem,
na prática, participarem ativamente das operações custosas do algoritmo Paxos. Alterar o
número de réplicas participantes de um sistema que usa replicação ativa não é uma tarefa
trivial pois a informação da \emph{cardinalidade} do conjunto de réplicas no sistema é
importante para o mecanismo de replicação gerenciar a consistência de estado da aplicação.
Isto ocorre pois toda alteração de estado deve ser votada e aprovada por uma maioria
destas réplicas. Assim, toda redução ou expansão neste conjunto deve ser precedida de
reconfiguração para que o número de votantes seja claramente definido.
A replicação de estado de forma síncrona recebe o nome de replicação ativa e consiste de
um conjunto de técnicas utilizadas para compartilhar ativa e completamente o estado de
processamento entre as réplicas.

Quando um servidor é retirado do sistema a aplicação deve ser reconfigurada para
trabalhar sem esta réplica. Isso afeta diretamente o mecanismo de verificação de conflitos
devido ao fato de existir $N-1$ potenciais eleitores do novo estado. Será criada uma nova
visão do sistema onde o servidor retirado não faz mais parte do processamento
computacional~\cite{lamport10}. A redução no número de máquinas potencialmente diminui o
poder computacional do aglomerado, tarefa que pode ser realizada quando é detectado
ociosidade (diminuição de requisições) no sistema, ou seja, o sistema está superestimado
para a demanda vigente.

Ao adicionar uma nova máquina a dificuldade é ainda maior, primeiro é necessário
transferir o estado atual da aplicação para o nova réplica, para que esta seja capaz de
votar no próximo estado da  aplicação. Após a tarefa de atualização de estado é preciso
reconfigurar o sistema incluindo o novo membro, criando uma nova visão com $N+1$
participantes. A expansão do aglomerado potencialmente aumenta seu poder computacional,
tarefa necessária quando é  detectado aumento de requisições, ou seja, o sistema está
subestimado para a demanda vigente.

Este comportamento  caracteriza, em sua forma mais simples, um comportamento elástico que
o sistema deve suportar para melhorar sua eficiência durante seu período de execução. Em
geral, sistemas de replicação ativa não suportam este mecanismo, sendo definidos
geralmente para grupos  de processos estáticos~\cite{chandra96, lamport98}. Do ponto de
vista teórico, podemos encontrar um tratamento formal do problema~\cite{lamport10}, mas
sistemas práticos tendem a evitar esta questão de forma a simplificar a construção do
sistema~\cite{chandra07}.


