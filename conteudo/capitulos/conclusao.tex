\chapter*[Conclusão]{Conclusão}\label{conclusao}
\addcontentsline{toc}{chapter}{Conclusão}

Nesse trabalho nós mostramos a implementação e os resultados experimentais de duas novas
funcionalidades adicionadas em Treplica: (1) mecanismo de transferência de estado; e (2)
réplicas leitoras. Conforme pudemos observar os resultados experimentais obtidos com o
mecanismo de transferência de estado motivam na expansão da utilização em réplicas
votantes, pois atualmente ele é exclusivo para réplicas leitoras.

O experimento de transferência de estado utiliza a adição de uma réplica votante e
leitora, embora seja injusto, é inevitável comparar qual réplica teve a melhor aderência
no sistema. Pudemos observar que a adição de uma réplica leitora gerou menos impacto no
desempenho do aglomerado. Esse bom desempenho na expansão do aglomerado não pode ser
atribuído para réplicas leitoras, mas sim ao protocolo de transferência. Esse é um
resultado interessante, pois réplicas votantes também podem ser beneficiar da utilização
do protocolo de transferência com o ônus de reconfiguração total \cite{lamport10} em
caso de expansão.

No entanto, alterar o número de réplicas participantes de um sistema que usa replicação
ativa não é uma tarefa trivial pois a informação da cardinalidade do conjunto de réplicas
do sistema é importante para o mecanismo de replicação gerenciar a consistência de estado
da aplicação. Isso ocorre pois toda alteração de estado deve ser votada e aprovada por uma
maioria destas réplicas. Assim, toda redução ou expansão neste conjunto deve ser precedida
de reconfiguração para que o número de votantes seja consistente entre as réplicas. É
baseado nesse cenário a maior vantagem que uma réplica leitora pode oferecer. Pudemos
observar que o desempenho de réplicas leitoras é parecido com o desempenho de réplicas
votantes, porém a adição de réplicas leitoras remove completamente a necessidade de
reconfiguração. Infelizmente ainda não temos como mensurar o impacto da reconfiguração
total em Treplica, por isso não temos uma resposta clara para a reponde a pergunta: é
preferível adicionar uma réplica votante ou leitora?

Acreditamos que uma maior escalabilidade pode ampliar as oportunidades de aplicação de
replicação ativa para novos contextos. Especificamente, aplicações Web com uma parcela
considerável de requisições de leitura podem se beneficiar da solução proposta.

Por fim, vale ressaltar a redução de desempenho de acordo com o aumento de requisições de
escrita. Embora uma requisição de escrita exija mais trocas de mensagens Paxos entre as
réplicas, pudemos notar que esse não era a maior razão para a redução de desempenho.
Observarmos um grande número de conflitos entre processos tentando propor alterações de
estado diferentes na mesma rodada. Paxos tolera apenas uma proposta por rodada, em
Treplica quando esse cenário é detectado, o agente coordenador envia uma mensagem para o
proponente que gerou conflito aprender o que já foi decidido na rodada conflitante. Sendo
assim, o proponente cria uma nova proposta com um número diferente de rodada e tenta
novamente estabelecer o consenso com a possibilidade de um novo conflito. Notamos que
conforme o aumento de propostas (ou agentes proponentes), maior é o número de conflitos.
Treplica utiliza um mecanismo descentralizado para geração de número de rodadas baseado
no conhecimento do estado local da réplica. Esse mecanismo não determinístico causa
problemas de conflito quando atua dessincronizado. Infelizmente a exploração desse
problema não faz parte do escopo desse trabalho.


\section*{Trabalhos futuros}

As propostas apresentadas por esse trabalho é um passo em direção ao mecanismo de
autogestão cobiçado para Treplica. Realizar autointegração de réplicas sem intervenção
humana exige criação de mecanismos para monitorar alterações no comportamento da aplicação
e infraestrutura \cite{renesse03, pierre06}. A partir do momento que o sistema está ciente
das mudanças de comportamento, pode tomar decisões para iniciar réplicas temporárias,
quando existe aumento de demanda ou até desligá-las após detectar queda no número de
acessos. Essa ideia de dimensionamento elástico leva a uma pergunta: quantas réplicas são
necessárias para suportar uma determinada carga de maneira eficiente? A resposta dessa
pergunta é o coração da pesquisa para criação de um mecanismo de monitoração em Treplica.

Outro ponto não atacado por esse trabalho é a implementação de um mecanismo híbrido de
detecção de lacunas no estado. A partir da suposição que, para uma lacuna pequena, o
mecanismo de transferência de de estado não é menos eficiente que a retransmissão de
instâncias de consenso, esse mecanismo seria capaz de escolher o melhor preenchedor de
lacunas de acordo com a condição do estado. Dessa forma, Treplica poderia suportar de
maneira mais eficiente falha-e-recuperação nas réplicas, além de expandir a utilização do
protocolo de transferência de estados para réplicas votantes (completas).

Conforme descrito na \autoref{subsec:funcionamento_protocolo}, o protocolo de
transferência de estado utiliza uma conexão TCP ponto a ponto entre a réplica doadora e
receptora. Essa é uma variável não controlada que pode limitar o desempenho da
transferência de estado. Uma outra possível abordagem, não pertencente ao escopo desse
trabalho é a utilização de um mecanismo colaborativo para envio do estado, onde partes do
estado seriam enviados por diferentes réplicas doadoras, semelhante ao protocolo P2P.
Gostaríamos de saber qual será o impacto que essa abordagem pode causar no desempenho de
Paxos.

Gostaríamos também de ressaltar que o mecanismo de transferência de estado trás para
Treplica a possibilidade da coleta de lixo. Instâncias de consenso antigas podem ser
removidas da memória a partir do momento que nenhuma réplica está mais interessada no
resultado desse consenso. O resultado dessa abordagem é a utilização mais eficiente da
memória.

Conforme já afirmamos anteriormente, esse trabalhamo atua em um sub-conjunto do problema
de reconfiguração total descrito por \cite{lamport10}. Adotamos essa abordagem para
explorar outros mecanismos como: transferência de estado e réplicas leitoras, além de
fugir da complexidade da reconfiguração total. No entanto, pudemos observar nos resultados
dos experimentos que a utilização de réplicas leitoras é interessante mas não substituí
uma réplica votante. Esse trabalho também ajuda a esclarecer o que precisa ser feito para
uma reconfiguração total em Treplica.

\section*{Publicações}


\section*{Submissões}


