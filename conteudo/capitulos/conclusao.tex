\chapter*[Conclusão]{Conclusão}\label{sec:conclusao}
\addcontentsline{toc}{chapter}{Conclusão}

Nesse trabalho nós mostramos a implementação e os resultados experimentais de duas novas
funcionalidades adicionadas em Treplica: (1) mecanismo de transferência de estado; e (2)
réplicas leitoras.

Em relação ao mecanismo de transferência de estado pudemos observar que a adição de uma
réplica usando o mecanismo proposto gerou menos impacto no desempenho do aglomerado.
Dependendo do tamanho do estado a ser transferido os ganhos são consideráveis. Esse é um
resultado interessante, pois pois mecanismos de reconfiguração completos também podem se
beneficiar da utilização do protocolo de transferência de estado em caso de expansão com o
ônus de reconfiguração total \cite{lamport10}. Os resultados experimentais obtidos com o
mecanismo de transferência de estado motivam na expansão da utilização em réplicas
votantes, pois atualmente ele é exclusivo para réplicas leitoras.

No experimento de réplicas leitoras pudemos observar uma equivalência de desempenho entre
réplicas leitoras e réplicas votantes. Validamos que não existe um grande ganho de
desempenho quando utilizamos réplicas leitoras em um cenário configurado com um número
significativamente maior de requisições de leitura. Também é verdade, que no cenário
configurado com mais requisições de escrita as réplicas leitoras não foram o motivo do
menor desempenho. Pudemos concluir que réplicas leitoras não oferecem um grande ganho de
desempenho.

No entanto, alterar o número de réplicas participantes de um sistema que usa replicação
ativa não é uma tarefa trivial, como argumentamos na \autoref{sec:reconfiguracao}.
Contornar essa dificuldade é a maior vantagem que uma réplica leitora pode oferecer. A
equivalência de desempenho com réplicas votantes passa a ser um resultado satisfatório,
pois conseguimos não diminuímos o desempenho da aplicação ao adicionar uma réplica que
remove completamente a necessidade de reconfiguração. Acreditamos que uma maior
escalabilidade pode ampliar as oportunidades de aplicação de replicação ativa para novos
contextos. Especificamente, aplicações Web com uma parcela considerável de requisições de
leitura podem se beneficiar da solução proposta.

Acreditamos que construímos mecanismos úteis para sistemas de replicação ativa,
principalmente para Treplica que ganha um novo fôlego rumo a reconfiguração total.


\section*{Trabalhos futuros}

As propostas apresentadas por esse trabalho são um passo em direção ao mecanismo de
autogestão cobiçado para Treplica. Realizar autointegração de réplicas sem intervenção
humana exige criação de mecanismos para monitorar alterações no comportamento da aplicação
e infraestrutura \cite{renesse03, pierre06}. A partir do momento que o sistema está ciente
das mudanças de comportamento, pode tomar decisões para iniciar réplicas temporárias
quando existe aumento de demanda ou até desligá-las após detectar queda no número de
acessos. Essa ideia de dimensionamento elástico leva a uma pergunta: quantas réplicas são
necessárias para suportar uma determinada carga de maneira eficiente? A resposta a essa
pergunta é o coração da pesquisa para criação de um mecanismo de monitoração em Treplica.

Outro ponto não atacado por esse trabalho é a implementação de um mecanismo híbrido de
detecção de lacunas no estado. A partir da suposição que, para uma lacuna pequena, o
mecanismo de transferência de de estado não é menos eficiente que a retransmissão de
instâncias de consenso, esse mecanismo seria capaz de escolher o melhor preenchedor de
lacunas de acordo com a condição do estado. Dessa forma, Treplica poderia suportar de
maneira mais eficiente falha-e-recuperação nas réplicas, além de expandir a utilização do
protocolo de transferência de estados para réplicas votantes (completas).

Conforme descrito na \autoref{subsec:funcionamento_protocolo}, o protocolo de
transferência de estado utiliza uma conexão TCP ponto a ponto entre a réplica doadora e
receptora. Essa abordagem pode limitar o desempenho da transferência de estado na medida
que estamos limitados a vazão de uma única réplica doadora. Uma outra possível abordagem,
não pertencente ao escopo desse trabalho, é a utilização de um mecanismo colaborativo para
envio do estado, onde partes do estado seriam enviados por diferentes réplicas doadoras,
semelhante ao protocolo P2P. Gostaríamos de saber qual será o impacto que essa abordagem
pode causar no desempenho de Paxos.

Gostaríamos também de ressaltar que o mecanismo de transferência de estado trás para
Treplica a possibilidade da coleta de lixo. Instâncias de consenso antigas podem ser
removidas da memória a partir do momento que nenhuma réplica está mais interessada no
resultado desse consenso. Como na presença de falhas não é possível ter certeza que uma
instância não será mais necessária para uma réplica, temos que fazer uma coleta de lixo
otimista descartando instâncias com fortes indícios de serem desnecessárias. Em caso de
erro, será necessário fazer uma transferência de estado para suprir na réplica que depende
de uma instância descartada o estado perdido. O resultado dessa abordagem é a utilização
mais eficiente da memória.

Esse trabalhamo considerou um sub-conjunto do problema de reconfiguração total descrito
por \citeonline{lamport10}. Adotamos essa abordagem para explorar outros mecanismos como
transferência de estado e réplicas leitoras, além de fugir da complexidade da
reconfiguração total. No entanto, pudemos observar nos resultados dos experimentos que a
utilização de réplicas leitoras é interessante mas não substituí uma réplica votante. Esse
trabalho também ajuda a esclarecer o que precisa ser feito para uma reconfiguração total
em Treplica.

Por último o problema de conflitos nas instâncias de Paxos descrito na
\autoref{subsec:funcionamento_protocolo} requer uma investigação mais detalhada. Uma
proposta de mecanismo para resolução de conflitos beneficiará tanto réplicas tradicionais
quanto leitoras e pode garantir ainda mais o aumento de escalabilidade no aglomerado que
utiliza Paxos como meio de replicação ativa.


\section*{Publicações}

Tornando Paxos Mais Escalável com Réplicas Leitoras. Em \emph{Anais do XIII Workshop em
Desempenho de Sistemas Computacionais e de Comunicação (WPerformance 2014)}, páginas
2014-2018. Evento organizado pelo Congresso da Sociedade Brasileira de Computação (CSBC)
em Julho de 2014 na cidade de Brasília.

