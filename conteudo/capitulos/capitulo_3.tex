\chapter{Avaliação e Resultados}\label{cap3}

O \autoref{cap1:treplica_reconfiguravel} apresentou três funcionalidades para expansão da
biblioteca Treplica:

\begin{enumerate}
  \item Protocolo para transferência de estado: criação de um mecanismo eficiente para
    transferência de estado entre réplicas.
  \item Réplicas leitoras: possibilidade da utilização de réplicas que não participam do
    processo de decisão de instâncias de consenso.
  \item Equalização de estado: proposta para novo componente de preenchimento de lacunas
    originadas por possíveis períodos de instabilidade da réplica e/ou falhas.
\end{enumerate}

Supomos duas hipóteses baseado nas alterações propostas 1 e 2. A alteração proposta pelo
item 3 não será validada por esse trabalho. Mantivemos a descrição dessa alteração para
enriquecimento e possível utilização em trabalhos futuros.

\begin{enumerate}
  \item Se possuirmos um mecanismo transferência eficiente, podemos recuperar o estado de
    uma réplica de forma mais eficiente que a aplicação do histórico de decreto do
    aglomerado. Lembrando que, toda réplica votante armazena em memória persistente
    qualquer alteração realizada no seu estado.
  \item Se possuirmos uma réplica que evite a reconfiguração total do sistema, ganharemos
    o poder de manobra necessário para expansão do aglomerado de acordo com a demanda de
    clientes.
\end{enumerate}

Para validar que a hipótese 1 e 2 podem aumentar o desempenho de um sistema que utiliza a
biblioteca Treplica, supomos uma carga de trabalho na qual uma parcela significativa das
requisições solicitadas para aplicação seja de leitura. Essa é uma suposição razoável para
a maioria das aplicações de Internet \cite{tpc02} e proporciona o cenário, que acreditamos
ser o mais adequado, para execução eficiente utilizando réplicas leitoras.

Desenvolvemos então, para esse conjunto de experimentos, uma aplicação Web simples que
mapeia uma cadeia de caracteres para um valor numérico de 32 bits. Essa aplicação
disponibiliza seus serviços aos clientes remotos através de uma interface HTTP. Começamos
esse capítulo com a \autoref{sec:aplicacao}, apresentando os detalhes e bibliotecas
utilizadas para concepção da aplicação. Em seguida, na \autoref{sec:ambiente_experimental}
descrevemos o ambiente experimental utilizado para execução dos experimentos. As próximas
duas seções \autoref{sec:experimento_tranferencia_extado} e
\autoref{sec:experimento:replicas_leitoras} apresentam respectivamente os experimentos de
transferência de estado e réplicas leitoras com seus respectivos cenários, resultados e
análise. Encerramos o capitulo com a \autoref{sec:trabalhos-relacionados} apresentando as
considerações finais.

\section{Aplicação}\label{aplicacao}


\section{Descrição do Ambiente Experimental}\label{sec:ambiente_experimental}

Aplicação, balanço carga, geração carga.

\section{Experimento Transferência de Estado}\label{sec:experimento_tranferencia_estado}

Cenários, carga, servidores

Resultados

Análise

\section{Experimento Elástico}\label{sec:experimento:replicas_leitoras}

Cenários, carga, servidores

Resultados

Análise

\section{Experimento Transferência de Estado}

Cenários, carga, servidores

Resultados

Análise


Descrever a proposta e sua implementação.

